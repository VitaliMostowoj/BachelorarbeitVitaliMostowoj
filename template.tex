%Standard Dokumenten Einstellungen
%Einstellen der Schriftgröße
% \documentclass[a4paper,12pt]{scrreport}
\documentclass[a4paper,12pt, parskip=full]{scrreprt}

% Eingabe der benötigten Konfigurationen
% !TeX root = ../template.tex

%Inhalt der Titelseite
\newcommand{\Titel}{Entwicklung einer Hardware-Health-Monitoring Lösung für Pepperl+Fuchs HMI Systeme}
\newcommand{\Art}{Bachelorarbeit}
\newcommand{\Vorname}{Vitali}
\newcommand{\Nachname}{Mostowoj}
\newcommand{\Studiengang}{Elektrotechnik}
\newcommand{\Abgabedatum}{018.09.2023}
\newcommand{\Bearbeitungszeitraum}{12 Wochen}
\newcommand{\Matrikelnummer}{9960312}
\newcommand{\Kurskrzl}{Tel20 At1}
\newcommand{\Ausbildungsfirma}{Pepperl + Fuchs SE}
%Benötigt bei Elektrotechnik Titelseite
\newcommand{\Abteilung}{HMI}
\newcommand{\Standort}{ Lilienthalstraße 200, 68307 Mannheim}
\newcommand{\BetreuerFirma}{Dr. Marc Seissler}
\newcommand{\BetreuerDHBW}{Prof. Dr. Joachim Priesnitz}

%Nur benötigt wenn Sperrvermerk verwendet wird
\newcommand{\SperrvermerkAuslaufDatum}{31.12.2222}

 
\input{config/packages.tex}


%===============================================================================
%Start des Dokuments
\begin{document}

%Die Gliederung entspricht den Richtlinien der Informationstechnik Fakultät, bitte anpassen für die jeweiligen Richtlinien

%Titelblatt
\include{sections/defaults/title/title.tex}

%List of Todos
\listoftodos\clearpage

%Sperrvermerk
\pagestyle{empty}
% !TeX root = ../../../template.tex
\chapter*{Sperrvermerk}
\glqq Der Inhalt dieser Arbeit darf weder als Ganzes noch in Auszügen Personen außerhalb des Prüfungsprozesses und des Evaluationsverfahrens zugänglich gemacht werden, sofern keine anders
lautende Genehmigung des Dualen Partners vorliegt.\grqq{} [Ende der Sperrfrist: \SperrvermerkAuslaufDatum]

\vspace{1.5cm}

\begin{tabularx}{0.9\textwidth}[b]{p{7cm} X p{7cm}}
\cline{1-1} \cline{3-3}
Ort, Datum &  & Unterschrift
\end{tabularx}
% \include{sections/defaults/blocking-notice/blocking-notice-et.tex}

\pagenumbering{Roman}

%Eigenleistung
\include{sections/defaults/personal-contribution/personal-contribution.tex}

%Abstract
% !TeX root = ../../template.tex
\chapter*{Abstract}
Die Firma \ac{p+f} ist im Bereich der Prozessautomation führender Hersteller für industrielle Sicherheitsausstattung. Das Produktportfolio umfasst eine Reihe industrieller Computersysteme für den Einsatz in explosionsgeschützten Bereichen. Durch den Einsatz in industriellen Umgebungen, sind diese Systeme Umwelteinflüssen wie Sonneneinstrahlung oder Vibrationen ausgesetzt. Durch eine falsche Abschätzung dieser Umwelteinflüsse können schädliche und unzulässige Betriebe entstehen. Oftmals können diese Betriebe nicht wahrgenommen werden. Um diesem Problem entgegenzuwirken, soll die in den Systemen verbaute Sensorik verwendet werden, um auf solche Betriebe hinzuweisen.\\
Diese Arbeit befasst sich im ersten Teil mit der Erarbeitung eines Konzepts, so wie einer Architektur für eine solche Healthmonitoring Lösung für die \acl{p+f} \ac{hmi}-Plattformen. Dabei wird eine Architektur entwickelt, welche es ermöglicht plattformunabhängig Sensorwerte auszulesen und zu speichern. Des Weiteren wird ein Modell zur Zustandsbewertung der Plattformen definiert.\\
Im zweiten Teil dieser Arbeit wird ein Prototyp der Anwendung implementiert. Dieser liest Sensoren der Plattformen aus, bewertet anhand der ausgelesenen Daten den Zustand der Geräte und stellt die Daten einer Netzwerk-Schnittstelle bereit.
Zur Visualisierung der Daten wir zudem ein Dashboard erstellt, welches dezentral auf einem separaten Rechner bereitgestellt wird. In diesem können Sensordaten der Plattformen in Echtzeit abgerufen und präsentiert werden. 
% % !TeX root = ../../template.tex
\chapter*{Abstract}
The company \ac{p+f} is a leading manufacturer of industrial safety equipment in the field of process automation. The product portfolio includes a range of industrial computer systems for use in explosion-proof areas. Due to their use in industrial environments, these systems are exposed to environmental influences such as solar radiation or vibrations. An incorrect assessment of these environmental influences can result in harmful and unacceptable operations. Often these operations cannot be perceived. To counteract this problem, the sensor technology installed in the systems is to be used to indicate just such operations.\\
The first part of this thesis deals with the development of a concept and an architecture for such a health monitoring solution for the \acl{p+f} \ac{hmi} platforms. Thereby, an architecture is created, which allows platform-independent reading and storing of sensor values. Furthermore, a model for the state evaluation of the platforms is defined.
In the second part of this thesis a prototype of the application is implemented. It reads sensors of the platforms, evaluates the state of the device based on the read data and provides the data to a network interface.
For the visualization of the data a dashboard is created, which is provided decentralized on a separate computer. In this dashboard, sensor data from the platforms can be retrieved in real time.

%===================================================
%Verzeichnisse: Verwendete Verzeichnisse Aktivieren durch entfernen von: % vor dem \

%Inhaltsverzeichnis
\addcontentsline{toc}{chapter}{Inhaltsverzeichnis}\tableofcontents\clearpage

%Abbildungsverzeichnis
\addcontentsline{toc}{chapter}{Abbildungsverzeichnis}\listoffigures\clearpage

%Tabellenverzeichnis
\addcontentsline{toc}{chapter}{Tabellenverzeichnis}\listoftables\clearpage

%Lstingverzeichnis
% \addcontentsline{toc}{chapter}{Listingverzeichnis}\lstlistoflistings\clearpage

%Abkürzungsverzeichnis
% Wichtig, damit man nicht immer von Vorlage.tex bauen muss
% !TEX root = ../Vorlage.tex

\chapter*{Abkürzungsverzeichnis}
\begin{acronym}[slmtA]
    \acro{p+f}[P+F]{Pepperl+Fuchs}
    \acro{mtbf}[MTBF]{Mean Time Between Failures}
    \acro{hmi}[HMI]{Human-Machine-Interface }
    \acro{sql}[SQL]{Structured Query Language}
    \acro{sequel}[SEQUEL]{Structured Englisch Query Language}
\end{acronym}

%===============================================================================

%Vorwort: Um Vorwort zu verwenden, % entfernen
% \include{sections/other/preface.tex}

\pagestyle{scrheadings}
\pagenumbering{arabic}
%===============================================================================

%Eigene Kapitel
%Kapitel als eigene .tex datei erstellen und einbinden mit: \include{Pfad/Dateiname}
%\include{sections/example}
\chapter{Einleitung}
\section{Pepperl + Fuchs / HMI}
Die \ac{p+f} wurde 1945 von Walter Pepperl und Ludwig Fuchs gegründet. Anfangs war sie eine Radioreperaturwekstadt, welche sich erst nach der Entwicklung eines eigenen Näherungsschalters so wie eines eigensicheren Transistorverstärkers auf das gebiet der Elektronik ausweitete. Inzwischen entwickelt, produziert und vertreibt \ac{p+f} Baugruppen und Sensoren für den Automatisierungsmarkt.\\
\begin{flushleft}
    \begin{figure}[h!]
        \centering
        \includegraphics[scale = 0.35]{P+F_Standorte.png}
        \caption{Standorte der Pepperl+Fuchs SE}
        \label{fig:StandortePF}
    \end{figure}
\end{flushleft}
Pepperl\todo{mehr auf HMI eingehen, warum braucht man diese, wo werden die eingesetzt}
Im Bereich der Prozessautomation ist \ac*{p+f} führender Hersteller industrieller Sicherheitsausstattungen. Das Produktportfolio umfasst Trennbarrieren, Signaltrenner, Zener-Barrieren, Feldbus-Technologien, Remote-I/O, HART-Interface-Lösungen, Mensch-Maschine-Schnittstellen (HMI) für Gefahrenbereiche, Füllstandsüberwachung, Überdruckkapselungssysteme, Schaltschränke,Feldverteiler und Warnsysteme für Ex-Umgebungen.\\
Die Abteilung \textit{Human Machine Interfaces} beschäftigt sich dabei mit der Entwicklung von Industrieller Computer Hard- und Software. Dies reicht von zubehör bishin zu vollständige Systemen mit Ex-Umgebungen Zertifizierung. Ziel der Systeme ist die Überwachung und Steuerung sämtlicher Produktionsschritte.   

\section{Problemstellung und Ziel der Arbeit}

\section{Anforderungen}
%\chapter{Stand der Technik}

\section{Vergleich von Computer Monitoring Software auf dem Markt}
\subsection{Computer Monitoring Software}
\subsection{HWiNFO}

\section{VisuNetHardware}
\subsection{VisuNet FLX}
\subsection{VisuNet GXP}

\newpage

\section{Software Design Konzepte}
Eine solide Softwarerchitektur ist entscheidend für die erfolgreiche Entwicklung und Wartung eines Programmes. Sie legt den Grundstein für die anschließende Implementierung. Durch eine gute Architektur wird sichergestellt das das Programm Skalierbar, Effizient, Robust und gut zu warten ist.\\
Hierbei bieten sogenannte Design Patterns abhilfe. \textit{Jedes Muster beschreibt zunächst ein in userer Umwelt immer wieder auftretendes Problem, beschreibt dann den Kern der Lösung dieses Problems, und zwar so dass man diese Lösung milionenfach anwendden kann, ohne sich je zu wiederholen} (Christof Alexander \textit{Eine Muster-Sprache} [Löcker verlag, Wien, 1995, Seite x]). Diese definition für muster bezieht sich auch auf objektorientierte Design Patterns. Das verwenden dieser Patterns ermöglicht Entwicklern von der Erfahrung anderer zu profitieren, um bereits gelöste Probleme nicht nochmal lösen zu Müssen. Zudem steigern sie auch die Codequalität. Der Code wird Lesbarer und die Wartung dessen wird leichter. Zudem wird auch die Implementierung neuer Erweiterungen und das Eindenken in die Software durch gängige Designpatterns erleichtert. \cite[S.25 ff]{DesignPatterns}\\
\textit{Alle gut strukturierten objektorientierten Architekturen basieren auf Mustern} (Grady Booch \cite[S.21]{DesignPatterns}).
In den folgenden Kapiteln wird genauer auf die in dieser Arbeit verwendeten Design Patterns eingegangen.      

\subsection{Adapter Pattern}
Zweck des Adapter Patterns ist die Anpassung der Schnitstelle einer Klasse an eine andere von dem Client erwarteten Schnitstelle. Somit ermöglicht das Pattern die Zusammenarbeit von zwei Klassen, welche auf grund ihrer Schnellen nicht möglich wäre. Das Adapter Patern ist auch unter dem namen Wrapper bekannt, welcher im folgenden Verlauf der arbeit verwendet wird.\\
Das Pattern kommt immer dann zum einsatz, wenn eine bereits existierende Klasse genutzt werden soll, jedoch die Schnitstelle der klasse nicht mit den aktuellen Anforderungen des clients übereinstimmt. Desweiteren wird das Dattern verwendet, wenn eine wiederverwendbare Klasse erzeugt werden soll, welche mit unabhängigen und nocht vorhersehbaren Klassen interagieren soll.\\      
\begin{center}
    \begin{figure}[h]
     \centering
     \includegraphics[width=1\linewidth]{UMLAdapterPattern}
     \caption{Adapter Pattern Struktur \cite{DesignPatterns}}
    \label{fig:AdapterPattern}
    \end{figure}
\end{center}
\vspace{-2cm}
Das Design Pattern besteht aus einem \textit{Target}, welches die vom Client verwendete Schnitstelle definiert. Zu dem kommt der \textit{Client}, welcher mit den Objekten zusammen arbeitet, die der Zielschnittstelle entsprechen. Zuletzt beinhaltet das Adapter Pattern einen \textit{Adaptee} so wie den \textit{Adapter} selbst. Der \textit{Adaptee} definiert eine bestehende Schnittstelle, welche vom \textit{Adapter} adaptiert werden muss.\\ 
Der \textit{Client} ruft die gewünschte Operation auf einer \textit{Adapter}-Instanz auf, welche anschließend die gewünschten \textit{Adaptee}-Operation ausführt.

\subsection{Strategie Design Pattern}
Zweck des Strategy (Strategie) Patterns ist es, eine Familie von einzelnen gekapselten und Austauschbaren Algorithmen zu schaffen. Dieses Patern ermöglicht eine variable und vom Client unabhöngige nutzung des Algorythmus.\\
Das Pattern kommt zum einsatz wenn eine Reihe von zusammenhängenden Klassen sich nur in Ihrem verhalten unterscheiden, verschiedene varianten eines Algorythmus erfordert werden, der Client keine Kenntnis von den vom Algorythmus verwendeten Daten haben soll, oder eine Klasse verschiedene Verhaltensweisen aufweist.\\
\begin{center}
    \begin{figure}[h]
     \centering
     \includegraphics[width=1\linewidth]{UMLStrategyPattern}
     \caption{Strategie Pattern Struktur \cite{DesignPatterns}}
    \label{fig:StrategyPattern}
    \end{figure}
\end{center}
\vspace{-2cm}
Das Design Pattern besteht aus den folgenden Teilnehmern. Die \textit{Strategy}, welche eine gemeinsame Schnitstelle für die verwendeten Algorithmen deklariert. Einer oder mehreren \textit{ConcreteStrategy}, welche die Implementierung der Algorythmen oder Klassen ist, so wie dem \textit{Context}, welcher mit eier \textit{ConcreteStrategy} ausgestattet wird. Desweiteren bestitzt der \textit{Context} eine Referenz auf das \textit{Strategy} Objekt.\cite[S.383 ff]{DesignPatterns}\\
Über den \textit{Context} kann anschließend zur laufzeit des Programmes die benötigten \textit{ConcreteStrategy} geladen und ausgeführt werden. 
Ein konkretes Beispiel hierzu wird im buch \cite[Head First Design Patterns]{HeadfirstDesignPatterns} behandelt, was den nutzen dieses Patterns nochmal verdeutlicht. 

\section{Datenbanken}
Weltweit wurden im Jahr 2022 Daten im Umfang von 103.66 Zettabyte erfasst. Diese Zahl wird sich laut Statistik \ref{fig:DatenvolumenStatistik} bis zum Jahr 2026 verdoppelt haben. Angesicht dieser Zahlen, sind Datenbanken aus der heutigen Zeit nicht weg zu denken. Sie bieten eine Möglichkeit, große Mengen an Daten Strukturiert abzuspeichern und anschließend auszuwerten.\\
Hierbei werden Datenbanken Grundsätzlich in Zwei Kategorien unterteilt. Relatione Datenbank und "Nicht relatione Datenbanken". Unterschiede der Datenbankarten machen sich in der Sprache zum Auswerten der DB, ihrer Skalierbarkeit, der Struktur, der Eigenschaften und der Unterstützung durch die Comunity bemerkbar. 
\begin{center}
    \begin{figure}[h]
     \centering
     \includegraphics[width=1\linewidth]{DatenvolumenStatistik}
     \caption{Volumen der weltweit generierten Daten bis 2027 \cite{Datenmengen}}
    \label{fig:DatenvolumenStatistik}
    \end{figure}
\end{center}
\subsection{SQL - Structured Query Language}
IBM-Forscher Edgar F. Codd definierte 1969 ein Datenbankmodell für Relationale Datenbanken. Auf grundlagen seiner Forschung began, in den folgenden Jahren, die entwicklung der Sprache \ac{SEQUEL}. Codds Modell für bassiert auf der zuordnung von Schlüsseln. Nach einigen Überarbeitungen der implementierung wurde diese anschließend in \ac{SQL} umbenannt.\\
\ac{SQL} ermöglicht insbesondere die Speicherung, Bearbeitung so wie eine Abfrage von Daten in einer Datenbank. Mithilfe des Prinzips der Schlüssel, können Datensätze miteinander verknüpft werden. Somit kann einem Benutzernamen bespielsweise ein echter Name, eine Telefonnummer und eine Email-Addresse zugewiesen werden.\\
Die besondere eigenschaft von \ac{SQL} ist das Konzepte von Arrays. Relationale Datenbanken bestehen aus Arrays, welche sich mit Hilfe von verschiedenen Befehen erzeugen und bearbeiten. \cite{SQL}\\
\ac{SQL} beitet eine reihe von Befehlen, welche die Interaktion mit der Datenbank ermöglichen. Diese können Grundsätzlich in 5 Kategorien eingeteilt werden (siehe Abb. \ref{fig:SQLCommands}). Die wichtigsten Befehle sind dabei \text{INSERT}, \textit{UPDATE} und \textit{DELEAT}, mit welchen sich datensätze schreiben und bearbeiten lassen. Zudem der kommt der \textit{SELECT} Befehl, welcher das auslesen von Datensätzen ermöglicht. Um die tabellenstruktur der Datenbank zu berarbeiten kommen die Befehle \textit{CREATE} und \textit{DROP} zum einsatz. \cite{SQLCommands}\\
Natürlich bietet die Programmiersprache eine weit aus komplexere Sysntax, um datensätze sortiet auswerten zu könen. Eine vollständige dokumentation der Sprache findet sich auf der w3school webseite \cite{SQLDoku}.
\begin{center}
    \begin{figure}[h!]
     \centering
     \includegraphics[scale = 0.3]{SQLCommands}
     \caption{SQL Befehls Kategorien \cite{SQLCommands}}
    \label{fig:SQLCommands}
    \end{figure}
   \end{center}

\subsection{SQLite Embedded Datenbank}
In der Vorarbeit zu dieser Bachelorarbeit wurde bereits eine auswahl für eine Datenbank getroffen. Dabei wurde sich nach einigen vergleichen für die SQLite Embedded Datenbank Engine entschieden. \\
Diese Bietet eine zuverlässige, kleine, schnelle und vollfunktionale Datenbank Engine, welche vollständige in das Gesamtsystem integriert werden kann \cite{SQLiteHompage}. Zur implementierung der Datenbank in die anwendung wird edie System.Data.SQLite bibliothek für C\# verwendet.\\

\section{MTBF und Reliability}


\section{Fuzzy Logic}

\section{Grafana}

%Grundlagen Kappitel
\chapter{Stand der Technik}
\section{Vorarbeiten zur Bachelorthesis}
Zum Thema dieser Bachelorarbeit wurden bereits zwei Vorarbeiten geleistet. Zum einen wurde im Rahmen einer Praxisphase, eine Voruntersuchung zum Thema \textit{Condition-Based-Monitorig für industrielle PCs}  vorgenommen. Die im Rahmen der Arbeit \cite{PAMathias} durchgeführte Grundlagenuntersuchung und Marktrecherche hat auf zwei Computer Monitoring Technologien aufmerksam gemacht. Diese wurden anschließend in der zweiten Vorarbeit \cite{t3000} evaluiert. Aus dieser Bewertung heraus, wurde sich für die \textit{HWiNFO} Software, zum Auslesen der auf der Hardware verbauten Sensoren entschieden. Die Software wird genauer in Abschnitt \ref{sec:HWiNFO} behandelt. Des Weiteren wurde in der Vorarbeit \cite{t3000} auch eine geeignete Datenbank für das Health Monitoring System ausgewählt. In Abschnitt \ref{sec:SQLite} wird die ausgewählte Datenbanktechnologie genauer beschrieben. 

\subsection{SQLite Embedded Datenbank}\label{sec:SQLite}\todo{Ergebnistabelle der Vorarbeit}
In der Vorarbeit zu dieser Bachelorarbeit wurde bereits eine Auswahl für eine Datenbank getroffen. Hierbei wurden drei Datenbanken in den Punkten Performanz, Größe der Anwendung, Ressourcennutzung und der Dokumentation miteinander verglichen. Die Resultate wurden in der Entscheidungs Matrix \ref{fig:MatrixDB} aufgeführt. Aus dem Vergleich  hervorgehend, wurde sich anschließend für die Verwendung der SQLite Embedded Datenbank Engine entschieden. Diese bietet eine zuverlässige, kleine, schnelle und voll funktionale Datenbank Engine, welche vollständige in das Gesamtsystem integriert werden kann \cite{SQLiteHompage}. Zur Implementierung der Datenbank in die Anwendung wird die System.Data.SQLite Bibliothek für C\# verwendet.\\ 
Auf das Thema Datenbanken wird in Abschnitt \ref{sec:Datenbank} genauer eingegangen.
\begin{flushleft}
    \begin{figure}[h!]
        \centering
        \includegraphics[width=1\linewidth]{MatrixDB}
        \caption{Entscheidungs Matrix zur Bewertung der Datenbanken \cite{t3000}}
        \label{fig:MatrixDB}
    \end{figure}
\end{flushleft}
\subsection{HWiNFO}\label{sec:HWiNFO}
\textit{Hwinfo} ist eine Software der Firma REALiX, welche zum Überwachen und Analysieren der Hardware eines Computers konzipiert wird \cite{HWINFO}. Über die grafische Oberfläche des Programms, lassen sich alle gesammelten Daten anzeigen. Der Nutzer kann diese Informationen nutzen, um Defekte an der Hardware zu erkennen.\\
Das ausschlaggebende Argument für die Nutzung der Software liegt in der \ac{api}. Über die Shared Memory Funktion der Software lassen sich alle Gerätedaten, die die Software auslesen kann über eine C\# Bibliothek auslesen. Hierzu muss die Funktion in den Einstellungen des Programms eingeschaltet werden. Da die 64 bit Version des Tools dies nur für einen Zeitraum von 12h erlaubt, wird für den Verlauf der Arbeit die 32-bit Version der Software verwendet. Zum anderen bietet der REALiX einen \ac{sdk} welcher alle Funktionen der Software in Form einer Bibliothek bereitstellt.\cite{HWINFO}\\
Im Verlauf der Arbeit wird die Shared Memory Funktion der Software für die prototypische Implementierung des Health Monitorig Systems genutzt.   
\section{Produktfamilie VisuNet}
Die Zielhardware für das Healthmonitoring System sind die \ac{p+f} VisuNet FLX und GXP Platformen. Diese sind Bedien- und Beobachtungssystemen für den Einsatz in Explosionsgefährdete bereichen. Durch  

\subsection{VisuNet FLX \& GXP}

\subsection{VisuNet RM Shell \& Control Center}
\section{Software Design Konzepte}
Eine solide Softwarerchitektur ist entscheidend für die erfolgreiche Entwicklung und Wartung eines Programmes. Sie legt den Grundstein für die anschließende Implementierung. Durch eine gute Architektur wird sichergestellt das das Programm Skalierbar, Effizient, Robust und gut zu warten ist.\\
Hierbei bieten sogenannte Design Patterns abhilfe. \textit{Jedes Muster beschreibt zunächst ein in userer Umwelt immer wieder auftretendes Problem, beschreibt dann den Kern der Lösung dieses Problems, und zwar so dass man diese Lösung milionenfach anwendden kann, ohne sich je zu wiederholen} (Christof Alexander \textit{Eine Muster-Sprache} [Löcker verlag, Wien, 1995, Seite x]). Diese definition für muster bezieht sich auch auf objektorientierte Design Patterns. Das verwenden dieser Patterns ermöglicht Entwicklern von der Erfahrung anderer zu profitieren, um bereits gelöste Probleme nicht nochmal lösen zu Müssen. Zudem steigern sie auch die Codequalität. Der Code wird Lesbarer und die Wartung dessen wird leichter. Zudem wird auch die Implementierung neuer Erweiterungen und das Eindenken in die Software durch gängige Designpatterns erleichtert. \cite[S.25 ff]{DesignPatterns}\\
\textit{Alle gut strukturierten objektorientierten Architekturen basieren auf Mustern} (Grady Booch \cite[S.21]{DesignPatterns}).
In den folgenden Kapiteln wird genauer auf die in dieser Arbeit verwendeten Design Patterns eingegangen.      

\subsection{Adapter Pattern}
Zweck des Adapter Patterns ist die Anpassung der Schnitstelle einer Klasse an eine andere von dem Client erwarteten Schnitstelle. Somit ermöglicht das Pattern die Zusammenarbeit von zwei Klassen, welche auf grund ihrer Schnellen nicht möglich wäre. Das Adapter Patern ist auch unter dem namen Wrapper bekannt, welcher im folgenden Verlauf der arbeit verwendet wird.\\
Das Pattern kommt immer dann zum einsatz, wenn eine bereits existierende Klasse genutzt werden soll, jedoch die Schnitstelle der klasse nicht mit den aktuellen Anforderungen des clients übereinstimmt. Desweiteren wird das Dattern verwendet, wenn eine wiederverwendbare Klasse erzeugt werden soll, welche mit unabhängigen und nocht vorhersehbaren Klassen interagieren soll.\\      
\begin{center}
    \begin{figure}[h]
     \centering
     \includegraphics[width=1\linewidth]{UMLAdapterPattern}
     \caption{Adapter Pattern Struktur \cite{DesignPatterns}}
    \label{fig:AdapterPattern}
    \end{figure}
\end{center}
\vspace{-2cm}
Das Design Pattern besteht aus einem \textit{Target}, welches die vom Client verwendete Schnitstelle definiert. Zu dem kommt der \textit{Client}, welcher mit den Objekten zusammen arbeitet, die der Zielschnittstelle entsprechen. Zuletzt beinhaltet das Adapter Pattern einen \textit{Adaptee} so wie den \textit{Adapter} selbst. Der \textit{Adaptee} definiert eine bestehende Schnittstelle, welche vom \textit{Adapter} adaptiert werden muss.\\ 
Der \textit{Client} ruft die gewünschte Operation auf einer \textit{Adapter}-Instanz auf, welche anschließend die gewünschten \textit{Adaptee}-Operation ausführt.

\subsection{Strategie Design Pattern}
Zweck des Strategy (Strategie) Patterns ist es, eine Familie von einzelnen gekapselten und Austauschbaren Algorithmen zu schaffen. Dieses Patern ermöglicht eine variable und vom Client unabhöngige nutzung des Algorythmus.\\
Das Pattern kommt zum einsatz wenn eine Reihe von zusammenhängenden Klassen sich nur in Ihrem verhalten unterscheiden, verschiedene varianten eines Algorythmus erfordert werden, der Client keine Kenntnis von den vom Algorythmus verwendeten Daten haben soll, oder eine Klasse verschiedene Verhaltensweisen aufweist.\\
\begin{center}
    \begin{figure}[h]
     \centering
     \includegraphics[width=1\linewidth]{UMLStrategyPattern}
     \caption{Strategie Pattern Struktur \cite{DesignPatterns}}
    \label{fig:StrategyPattern}
    \end{figure}
\end{center}
\vspace{-2cm}
Das Design Pattern besteht aus den folgenden Teilnehmern. Die \textit{Strategy}, welche eine gemeinsame Schnitstelle für die verwendeten Algorithmen deklariert. Einer oder mehreren \textit{ConcreteStrategy}, welche die Implementierung der Algorythmen oder Klassen ist, so wie dem \textit{Context}, welcher mit eier \textit{ConcreteStrategy} ausgestattet wird. Desweiteren bestitzt der \textit{Context} eine Referenz auf das \textit{Strategy} Objekt.\cite[S.383 ff]{DesignPatterns}\\
Über den \textit{Context} kann anschließend zur laufzeit des Programmes die benötigten \textit{ConcreteStrategy} geladen und ausgeführt werden. 
Ein konkretes Beispiel hierzu wird im buch \cite[Head First Design Patterns]{HeadfirstDesignPatterns} behandelt, was den nutzen dieses Patterns nochmal verdeutlicht.
\section{Datenbanken}\label{sec:Datenbank}
\begin{wrapfigure}{r}{0.5\textwidth}
    \vspace{-1.2cm}
    \captionsetup{justification=centering,format=plain, font=small}
    \begin{center}
      \includegraphics[width=0.48\textwidth]{DatenvolumenStatistik}
    \end{center}
    \vspace{-0.5cm}
    \caption{Volumen der weltweit generierten Daten bis 2027 \cite{Datenmengen}}
    \label{fig:DatenvolumenStatistik}
    \vspace{-0.5cm}
  \end{wrapfigure}
Weltweit wurden im Jahr 2022 Daten im Umfang von 103.66 Zettabyte erfasst \cite{Datenmengen}. Diese Zahl wird sich laut Statistik \ref{fig:DatenvolumenStatistik} bis zum Jahr 2026 verdoppelt haben. Angesichts dieser Zahlen, sind Datenbanken aus der heutigen Zeit nicht mehr wegzudenken. Sie bieten eine Möglichkeit, große Mengen an Daten strukturiert abzuspeichern und auszuwerten.\\
Hierbei werden Datenbanken grundsätzlich in zwei Kategorien unterteilt. Relationale und nicht relationale Datenbanken. Unterschiede der Datenbankarten machen sich in der Sprache zum Auswerten der Datenbank, ihrer Skalierbarkeit, der Struktur, der Eigenschaften und der Unterstützung durch die Community bemerkbar. \cite{SQLNoSQL} 

\subsection{SQL - Structured Query Language}
\begin{wrapfigure}{r}{0.5\textwidth}
    \vspace{-1.2cm}
    \captionsetup{justification=centering,format=plain, font=small}
    \begin{center}
      \includegraphics[width=0.48\textwidth]{SQLCommands}
    \end{center}
    \vspace{-0.5cm}
    \caption{SQL Befehls Kategorien \cite{SQLCommands}}
    \label{fig:SQLCommands}
    \vspace{-0.5cm}
  \end{wrapfigure}
IBM-Forscher Edgar F. Codd definierte 1969 ein Datenbankmodell für Relationale Datenbanken. Auf Grundlagen seiner Forschung begann in den folgenden Jahren die Entwicklung der Sprache \ac{sequel}. Codds Modell basiert auf der Zuordnung von Schlüsseln. Nach einigen Überarbeitungen der Implementierung wurde diese anschließend in \ac{sql} umbenannt.\cite{SQL}\\
\ac{sql} ermöglicht insbesondere die Speicherung, Bearbeitung sowie eine Abfrage von Daten in einer Datenbank. Mithilfe des Prinzips der Schlüssel können Datensätze innerhalb einer Datenbank miteinander verknüpft werden. Somit kann einem Benutzernamen beispielsweise ein echter Name, eine Telefonnummer und eine E-Mail Adresse zugewiesen werden.\cite{SQL}\\
Die besondere Eigenschaft von \ac{sql} ist das Konzept von Arrays. Relationale Datenbanken bestehen aus Arrays, welche sich mithilfe von verschiedenen Befehlen erzeugen und bearbeiten lassen.\cite{SQL}\\
\ac{sql} bietet dabei eine Reihe von Befehlen, welche die Interaktion mit der Datenbank ermöglichen. Diese können grundsätzlich in 5 Kategorien eingeteilt werden (siehe Abb. \ref{fig:SQLCommands}). Die wichtigsten Befehle sind dabei \textit{INSERT}, \textit{UPDATE} und \textit{DELEAT}, mit welchen sich Datensätze schreiben und bearbeiten lassen. Zudem kommt der Befehl \textit{SELECT}, welcher das Auslesen von bestimmten Datensätzen ermöglicht. Um die Tabellenstruktur der Datenbank zu bearbeiten, kommen die Befehle \textit{CREATE} und \textit{DROP} zum Einsatz.\cite{SQLCommands}\\
Natürlich bietet die Programmiersprache eine weitaus komplexere Syntax, um Datensätze sortiert auswerten zu können. Eine vollständige Dokumentation der Sprache findet sich auf der w3school Webseite \cite{SQLDoku}.
\section{MTBF und Reliability}\label{sec:MTBF}
Es gibt viele Ursachen, welche zu einem Ausfall elektronischer Komponenten in einem System führen können. Laut dem technischen Bericht \cite{AIP} ist in 50\% der Fälle die Temperatur der Komponenten für einen Ausfall verantwortlich. Dies ist auf die unterschiedlichen thermischen Ausdehnungskoeffizienten der auf der Platine verwendeten Materialien zurückzuführen. Durch die unterschiedliche Ausdehnung der Bauteile und der Platine selbst, kommt es zu hohen Belastungen der Lötstellen. Während sich dieser Zyklus wiederholt, können Risse in den Verbindungen entstehen und sich ausbreiten. Diese können anschließend zu einem Bruch im elektrischen Stromkreis führen. \cite{AREPA_LifeExpectancy}\\
\begin{wrapfigure}{r}{0.65\textwidth}
    \vspace{-1.2cm}
    \captionsetup{justification=centering,format=plain, font=small}
    \begin{center}
      \includegraphics[width=0.64\textwidth]{BathTubCurve}
    \end{center}
    \vspace{-0.5cm}
    \caption{Bathtub Curve \cite{AREPA_LifeExpectancy}}
    \label{fig:BathTubCurve}
    \vspace{-1.5cm}
  \end{wrapfigure}
Die in Abbildung \ref{fig:BathTubCurve} abgebildete Bathtub-Kurve ist ein Konzept, welches zur Beschreibung der Lebensdauer elektronischer Komponenten verwendet wird. Die Kurve beschreibt eine mittlere Betriebsdauer zwischen Ausfällen. Sie weist drei Betriebsphasen auf.\\
In der ersten Phase, bekannt als \textit{Infant Mortality}, kommt es durch Konstruktions-, Produktions- und Werkstoffmängel häufig gleich zu Beginn des Betriebs zu Fehlern und Ausfällen. Geräte, die von diesen Problemen nicht betroffen sind, laufen meist zuverlässig durch die zweite Phase der Kurve, bekannt als \textit{Random Failures}. Hierbei kommt es nur deutlich seltener zu Ausfällen. Zum Ende der Lebensdauer kommt es, in der \textit{Wear-Out} Phase, durch Alterung und Verschleiß wieder vermehrt zu Ausfällen.\cite{AREPA_LifeExpectancy}\\ 
Der \ac{mtbf} ist dabei eine statistische Kennzahl, die den durchschnittlichen Zeitraum in Stunden angibt, der zwischen zwei aufeinanderfolgenden Ausfällen einer bestimmten Komponente, eines Systems oder eines Produkts verstrichen ist. Dieser weist zudem eine Temperaturabhängigkeit auf. Beispielsweise bei Kapazitäten kann im Durchschnitt gesagt werden: \textit{Eine Erhöhung der Betriebstemperatur um 10°C, führt zu einer Halbierung der Lebenserwartung}. Ein \ac{mtbf} von 100h sagt also aus, dass ein System im Durchschnitt, 100h laufen wird, bevor es zu einem Fehler kommt.\cite{MTBFReliability}\\
Die Zuverlässigkeit (Reliability) eines Gerätes hingegen ist als die Wahrscheinlichkeit definiert, mit der ein System seine beabsichtigte Funktion für einen festgelegten Zeitraum erfüllen wird. Hat ein System bei 100h eine Zuverlässigkeit von 0.8, so besteht eine 80\% Wahrscheinlichkeit, dass das System nach 100h noch funktioniert.\cite{MTBFReliability}\\
Die Zuverlässigkeit eines Systems kann über den \ac{mtbf} mit der Formel \ref{equ:Reliability} berechnet werden. Dabei ist zu beachten, dass man die Temperaturabhängigkeiten des \ac{mtbf} berücksichtigen muss.
\begin{align}
    && R(t) &= e^{-\frac{t}{\text{mtbf}}} &&
    \label{equ:Reliability} 
\end{align}

\vspace{-2cm}
\section{FuzzyLogic}\label{sec:FuzzyLogic}
\begin{wrapfigure}{r}{0.4\textwidth}
    \vspace{-1.1cm}
    \begin{center}
      \includegraphics[width=0.4\textwidth]{FuzzyBoolLogic}
    \end{center}
    \vspace{-0.5cm}
    \caption{Vergleich von Fuzzy Logic zu Boolescher Logik \cite{FuzzyLogicGeeks}}
    \label{fig:FuzzyCompare}
    \vspace{-1.5cm}
  \end{wrapfigure}
Fuzzy Logic, erstmals in den 1960er Jahren von Lotfi Zadeh an der University of California entwickelt, stellt einen innovativen Ansatz der Datenverarbeitung dar, der auf Wahrheitsgraden basiert. Im Gegensatz zur herkömmlichen Booleschen Logik, die sich auf binäre Zustände von 1 oder 0 bzw. wahr oder falsch stützt, zeichnet sich die Fuzzy Logic durch ihre Fähigkeit aus, die Vielschichtigkeit von Zwischenzuständen zu berücksichtigen.\cite{FuzzyLogicTechTarget}\\
Das zentrale Merkmal der Fuzzy-Logik besteht darin, unpräzise Argumentationsweisen zu modellieren, die eine bedeutende Rolle in der bemerkenswerten Fähigkeit des Menschen spielen, unter Bedingungen der Ungewissheit und Ungenauigkeit rationale Entscheidungen zu treffen (Siehe Abbildung \ref{fig:FuzzyCompare}). Diese Fähigkeit basiert auf unserem Vermögen, aus einem Wissensbestand, der ungenau, unvollständig oder nicht völlig zuverlässig ist, ungefähre Antworten auf Fragen abzuleiten. Anders als in klassischen logischen Systemen strebt die Fuzzy Logic danach, die Grauzonen zwischen klaren Kategorien zu erfassen und somit eine flexiblere und menschlichere Art der Datenverarbeitung zu ermöglichen. Dieser Ansatz hat Anwendungen in verschiedenen Bereichen gefunden, darunter Steuerungssysteme, künstliche Intelligenz, Entscheidungsfindung und mehr. \cite{LoftiFuzzyLogic}

\subsection{Architektur eines Fuzzy Logic Systems}
Ein Fuzzy Logic Systems kann in vier Module unterteilt werden. Jede dieser Komponenten spielt dabei eine entscheidende Rolle für das gesamte System. 
Abbildung \ref{fig:FuzzyLogicArchitektur} zeigt den Aufbau eines Fuzzy Logic Systems auf.\\
Die Umwandlung der Systemeingänge ist Aufgabe des \textit{Fuzzification} Moduls. Dabei werden exakten Werte in sogenannte Fuzzy-Sets umgewandelt. Die \textit{Inference Engine} bestimmt anschließend den Grad der Übereinstimmung des aktuellen Fuzzy-Sets in Bezug auf jede Regel und trifft eine Entscheidung darüber, welche Regeln gemäß dem Eingangsfeld ausgelöst werden sollen. Durch eine Kombination der ausgelösten Regeln wird anschließend eine Steuerungsaktion formuliert. Das \textit{Defuzzification} Model wird im Anschluss dazu verwendet, aus den durch die \textit{Interence Engine} erhaltenen Fuzzy-Setz, exakte Ausgangswerte zu erhalten.
Die \textit{Rule Base} umfasst eine Sammlung von Regeln und den \grqq{\textit{IF-THEN}} Bedingungen, welche im Vorfeld definiert und dem System bereitgestellt werden. Diese werden verwendet, um das Entscheidungssystem auf Grundlage von linguistischen Variablen zu steuern. \cite{FuzzyLogicGeeks}\\
\vspace{-1.5cm}
\begin{center}
    \begin{figure}[h]
     \centering
     \includegraphics[width=0.69\textwidth]{FuzzyLogicArchitektur}
     \caption{Architektur eines Fuzzy Logic Systems \cite{FuzzyLogicGeeks}}
     \label{fig:FuzzyLogicArchitektur}
    \end{figure}
   \end{center}


\section{Grafana}\label{sec:Grafana}
Grafana ist eine Open-Source-Software zur plattformübergreifenden Analyse von Metriken und Daten. Als Webservice gehostet, ermöglicht es die Visualisierung von Daten mithilfe von vorgefertigten Vorlagen für Tabellen und Graphen. Diese Software unterstützt verschiedene Datenquellen und kann problemlos direkt mit einer relationalen Datenbanken verbunden werden. Grafana erleichtert somit die präzise Darstellung und Interpretation von Informationen durch benutzerfreundliche Visualisierungstools. \cite{GrafanaWebsite}\\
\vspace{-1cm}
\begin{center}
    \begin{figure}[h]
     \centering
     \includegraphics[width=1\textwidth]{Gitlab_Dashboard.png}
     \caption{Grafana Dashboard Beispiel \cite{GrafanaWebsite}}
     \label{fig:FuzzyLogicArchitektur}
    \end{figure}
   \end{center}


\section{Datenerfassung}\label{sec:Datenerfassung}
Die Herausforderung beim Auslesen der Daten liegt darin, die plattformabhängigen Informationen in einem allgemeinen Datenmodell zu konsolidieren. Hierbei stammen die Daten aus verschiedenen Schnittstellen. Die Architektur muss in der Lage sein, sämtliche verfügbaren Sensordaten plattformunabhängig auszulesen, darunter beispielsweise Temperaturen und CPU-Auslastung. Zusätzlich sollte sie die Integration weiterer plattformspezifischer Hardwarekonfigurationen ermöglichen, ohne eine grundlegende Neustrukturierung des bestehenden Codes zu erfordern.\\
Das Speichern der ausgelesenen Messdaten soll in einem einheitlichen Format erfolgen, das unabhängig von der Hardwarekonfiguration der Zielplatform ist. Zudem ist eine klare und sinnhafte Struktur der Datenbank wichtig, da diese Performant und Erweiterbar sein muss.   

\subsection{Entwurf einer Architektur zum Auslesen der Systemhardware}
\subsection{Entwurf eines Datenbankmodells zum Speichern der Messwerte}

%===============================================================================
\pagenumbering{Roman}
%Literaturverzeichnis
\addcontentsline{toc}{chapter}{Literaturverzeichnis}\printbibliography[title=Literaturverzeichnis]
%Anhang

\end{document}
