% Wichtig, damit man nicht immer von Vorlage.tex bauen muss
% !TEX root = ../Vorlage.tex

\chapter*{Abkürzungsverzeichnis}
\begin{acronym}[slmtA]
    \acro{p+f}[P+F]{Pepperl+Fuchs}
    \acro{mtbf}[MTBF]{Mean Time Between Failures}
    \acro{api}[API]{Application Programming Interface}
    \acro{sdk}[SDK]{Software Development Kit}
    \acro{hmi}[HMI]{Human-Machine-Interface}
    \acro{sql}[SQL]{Structured Query Language}
    \acro{sequel}[SEQUEL]{Structured Englisch Query Language}
    \acro{tcu}[TCU]{Thin Client Unit}
    \acro{dpu}[DPU]{Display Unit}
    \acro{http}[HTTP]{Hypertext Transfer Protocol}
    \acro{json}[JSON]{JavaScript Object Notation}
    \acro{fls}[FLS]{Fuzzy-Logic-Sharp}
    \acro{orm}[ORM]{Object Relational Mapping}
\end{acronym}
