\chapter{Ausblick}
Die im Rahmen dieser Arbeit entwickelte Anwendung findet in der \ac{hmi}-Abteilung eine Reihe von Anwendungen. Vor allem im Hinblick auf das Testen der \ac{hmi}-Plattformen. Die Anwendung bietet eine flexible Möglichkeit unter Test stehende Gerätedaten zu überwachen. Hierzu kann das Dashboard der Anwendung nach Belieben konfiguriert werden, um beispielsweise alle relevanten Temperatursensoren anzeigen zu lassen.\\
Um diese Auswertemöglichkeiten der Anwendung umfangreicher ausschöpfen zu können, sollte die Anwendung in zwei Punkten verbessert werden. Zum einen kann die Abstraktion durch die \textit{SQLiteWrapper}-Klasse der Datenbank von einem \ac{orm}-Tool übernommen werden. Diese ermöglichen die Verwaltung von Datenbanken in objektorientierten Anwendungen.\cite{ormDB}\\
Des Weiteren könnte die \ac{api} der Anwendung erweitert werden, sodass die Konfiguration der Anwendung vollständig über diese erfolgt. Somit könnten die Einstellungen zur Laufzeit der Anwendung nach Bedarf angepasst werden.\\
Zudem können weitere Sensoren in die Systemstatusbewertung miteinbezogen werden, was die Bewertung noch präziser und zuverlässiger machen kann.\\
Darüber hinaus bietet diese Arbeit eine solide Grundlage für zukünftige Forschungsarbeiten und Entwicklungen. Insbesondere im Hinblick auf Machine Learning kann die Anwendung dazu verwendetet werden, Trainingsdaten für die Implementierung eines neuronalen Netzwerks zu sammeln. Hierzu können verschiedene Szenarien, in denen das Gerät beispielsweise in einem Klimaschrank auch hohe Betriebstemperaturen gebracht wird, nachgestellt werden. Mithilfe der Hardware-Health-Monitoring Anwendung können dabei die Betriebstemperaturen und Systemzustände aufgezeichnet werden. Werden genug solcher Szenarien aufgezeichnet, kann mithilfe der aufgezeichneten Daten ein Machine Learning Modell Trainiert werden, welches anschließend weitere Aufschlüsse über Systemzustand liefern kann.