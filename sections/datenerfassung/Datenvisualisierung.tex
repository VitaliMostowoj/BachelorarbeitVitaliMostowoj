\newpage
\section{Visualisieruing der Daten }\label{sec:Datenvisualisierung}

Die Visualisierung der Messwerte sowie der ermittelten Zustandsbewertung des Systems erfolgt mithilfe der Open-Source-Software Grafana, welches bereits in Abschnitt \ref{sec:Grafana} beschrieben wurde. Diese Software kann lokal auf der Zielplattform installiert und ausgeführt werden. Zur Darstellung der Daten steht dem Programm eine Samlung von Diagrammen und Anzeigen zur verfügung. 





Zur Darstellung der Daten aus der Datenbank in Grafana stehen zwei potenzielle Datenquellen zur Auswahl.




Zum einen kann die \textit{.db}-Datei, in der die eingebettete Datenbank gespeichert wird, direkt über die \textit{SQLite}-Datenquelle in das Dashboard integriert werden. Zum anderen ist es möglich, über die \textit{SimpleJson}-Datenquelle, anfragen an eine REST-API zu versenden, um die Antworten im Dashboard zu visualisieren.\\
Zunächst jedoch muss 

\subsection{Bereitstellung der Daten}
\subsection{Konzeptentwurf zur Datenvisualisierung}