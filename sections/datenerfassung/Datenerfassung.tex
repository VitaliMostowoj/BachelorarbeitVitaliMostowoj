\section{Datenerfassung}\label{sec:Datenerfassung}
Die Herausforderung beim Auslesen der Daten liegt darin, die plattformabhängigen Informationen in einem allgemeinen Datenmodell zu konsolidieren. Hierbei stammen die Daten aus verschiedenen Schnittstellen. Die Architektur muss in der Lage sein, sämtliche verfügbaren Sensordaten plattformunabhängig auszulesen, darunter beispielsweise Temperaturen und CPU-Auslastung. Zusätzlich sollte sie die Integration weiterer plattformspezifischer Hardwarekonfigurationen ermöglichen, ohne eine grundlegende Neustrukturierung des bestehenden Codes zu erfordern.\\
Das Speichern der ausgelesenen Messdaten soll in einem einheitlichen Format erfolgen, das unabhängig von der Hardwarekonfiguration der Zielplatform ist. Zudem ist eine klare und sinnhafte Struktur der Datenbank wichtig, da diese Performant und Erweiterbar sein muss.   

\subsection{Entwurf einer Architektur zum Auslesen der Systemhardware}
\subsection{Entwurf eines Datenbankmodells zum Speichern der Messwerte}
