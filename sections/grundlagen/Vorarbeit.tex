\chapter{Stand der Technik}
In diesem Kapitel werden die in dieser Arbeit Verwendeten Konzepte und Technologien beleuchtet. 

\section{Vorarbeiten zur Bachelorthesis}
Zum Thema dieser Bachelorarbeit wurden bereits zwei Vorarbeiten geleistet. Zum einen wurde im Rahmen einer Praxisphase, eine Voruntersuchung zum Thema \textit{Condition-Based-Monitorig für industrielle PC's}  vorgenommen. Die im Rahmen dieser Arbeit \cite{PAMathias} durchgeführte Grundlagenuntersuchung und Marktrecherche hat auf zwei Computer Monitoring Technologien aufmerksam gemacht. Diese wurden anschließend in der zweiten Vorarbeit \cite{t3000} evaluiert. Aus dieser Bewertung heraus, wurde sich für die \textit{HWiNFO} Software, zum auslesen der auf der Hardware verbauten Sensoren entschieden. Die Software wird genauer in Abschnitt \ref{sec:HWiNFO} behandelt. Desweiteren wurde in der Vorarbeit \cite{t3000} auch eine geeignete Datenbank für das Health Monitoring System ausgewählt. In Abschnitt \ref{sec:SQLite} wird die ausgewählte Datenbank Technologie genauer beschrieben.   

\subsection{SQLite Embedded Datenbank}\label{sec:SQLite}
In der Vorarbeit zu dieser Bachelorarbeit wurde bereits eine auswahl für eine Datenbank getroffen. Dabei wurde sich nach einigen vergleichen für die SQLite Embedded Datenbank Engine entschieden. \\
Diese Bietet eine zuverlässige, kleine, schnelle und vollfunktionale Datenbank Engine, welche vollständige in das Gesamtsystem integriert werden kann \cite{SQLiteHompage}. Zur implementierung der Datenbank in die anwendung wird die System.Data.SQLite Bibliothek für C\# verwendet.\\
Auf das Thema Datenbanken wird in Abschnitt \ref{sec:Datenbank} genauer eingegangen.

\subsection{HWiNFO}\label{sec:HWiNFO}
