\chapter{Stand der Technik}
In diesem Kapitel werden die in dieser Arbeit Verwendeten Konzepte und Technologien beleuchtet. 

\section{Vorarbeiten zur Bachelorthesis}
Zum Thema dieser Bachelorarbeit wurden bereits zwei Vorarbeiten geleistet. Zum einen wurde im Rahmen einer Praxisphase, eine Voruntersuchung zum Thema \textit{Condition-Based-Monitorig für industrielle PC's}  vorgenommen. Die im Rahmen dieser Arbeit \cite{PAMathias} durchgeführte Grundlagenuntersuchung und Marktrecherche hat auf zwei Computer Monitoring Technologien aufmerksam gemacht. Diese wurden anschließend in der zweiten Vorarbeit \cite{t3000} evaluiert. Aus dieser Bewertung heraus, wurde sich für die \textit{HWiNFO} Software, zum auslesen der auf der Hardware verbauten Sensoren entschieden. Die Software wird genauer in Abschnitt \ref{sec:HWiNFO} behandelt. Desweiteren wurde in der Vorarbeit \cite{t3000} auch eine geeignete Datenbank für das Health Monitoring System ausgewählt. In Abschnitt \ref{sec:SQLite} wird die ausgewählte Datenbank Technologie genauer beschrieben.   

\subsection{SQLite Embedded Datenbank}\label{sec:SQLite}
In der Vorarbeit zu dieser Bachelorarbeit wurde bereits eine auswahl für eine Datenbank getroffen. Hierbei wurden drei Datenbanken in den Punkten Performanz, größe der Anwendung, Ressourcen Nutzung und der Dokumentation miteinander verglichen. Aus dem Vergleich hervogehend wurde sich anschließend für die verwendung der SQLite Embedded Datenbank Engine entschieden. Diese Bietet eine zuverlässige, kleine, schnelle und vollfunktionale Datenbank Engine, welche vollständige in das Gesamtsystem integriert werden kann \cite{SQLiteHompage}. Zur implementierung der Datenbank in die anwendung wird die System.Data.SQLite Bibliothek für C\# verwendet.\\ 
Auf das Thema Datenbanken wird in Abschnitt \ref{sec:Datenbank} genauer eingegangen.

\subsection{HWiNFO}\label{sec:HWiNFO}
\textit{Hwinfo} ist eine Software der Firma REALiX, welche zum Überwachen und Analysieren der Hardware eines Computers konzepiert wird. Über die Grafische Oberfläche des Programms, lassen sich alle gesammelten Daten anzeigen. Der Nutzer Nutzer kann diese Informationen nutzen, um Defekte an der Hardware zu erkennen.\\
Das ausschlaggebende Argument für die Nutzung der Software liegt in der \ac{api}. Über die Shared Memory funktion der software lassen sich alle Gerätedaten die die Software auslesen kann über eine C\# Bibliothek auslesen. Hierzu muss die Funktion in den einstellunen des Programms eingescahlten werden. Da die 64 bit Version des Tools dies nur für einen Zeitraum von 12h erlaub, wird für den Verlauf der arbeit die 32-bit Version der Software verwendet. Zum anderen bietet der REALiX einen \ac{sdk} welcher alle Funktionen der Software in Form einer Bibliothek bereit stellt.\cite{HWINFO}\\
Im Verlauf der Arbeit wird die Shared Memory Funktion der Software für die Prototypische implementierung des Health Monitorig Systems genutzt.   