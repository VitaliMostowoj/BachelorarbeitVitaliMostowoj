\newpage
\section{HM.API}
Zur Implementierung der REST \ac{api} wird die \textit{HttpListener} Klasse verwendet. Diese Klasse stellt einen HTTP-Protokollistener zur verfügung, welcher auf die \ac{http}-Anfragen antwortet. \cite{HttpListener}\\
Die \textit{HttpListener} Klasse wird von der \textit{DataBaseAPI} Klasse im \textit{HM.API} Verzeichnis abstrahiert. Des weitern referenziert die \textit{DataBaseAPI} Klasse das in Abschnitt \ref{sec:HMDBServices} beschriebene \textit{HM.DBServices} Verzeichnis und verwendet hierbei die \textit{SQLiteWrapper} und \textit{ConfigurationReader} Klassen. Desweiteren wird die Property \textit{Targets} definiert, welche eine Liste von zur auswahl stehenden Datensätze repräsentiert. Die Instanzen der einzelnen Klassen werden anschließend im Construktor der Klasse erzeugt. (Siehe Listing \ref{lst:DataBaseAPIRef})
\begin{lstlisting}[caption={Property der DataBaseAPI Klasse}, label={lst:DataBaseAPIRef}]
    namespace HM.API
    {
        public class DataBaseAPI
        {
            private HttpListener listener;
            private readonly SQLiteWrapper Database;
            private Configuration config;
            List<Target> Targets;
            
            public DataBaseAPI(SQLiteWrapper db)
            {
                SetConfiguration();
                this.Database = new SQLiteWrapper(config.databaseFileName);
                InitTargets();
            }
\end{lstlisting}
Um die Platformunabhängigkeit des Systems beizubehalten wird in der \textit{InitTargets()} Funktion der Klasse die Targetliste, abhängig von der in der Konfigurations Datei gesetzten Platform erstellt. Hierbei wird ein \textit{Target} Objekt an die \textit{Targets} Liste hinzugefügt. Dieses setzt sich aus den Parametern \textit{dwReadingID}, \textit{szLabelUser} und dem \textit{tableName}, um den richtigen Datensatz in der Datenbank ausfündig zu machen.\\
Sollen weitere Daten des Hardware-Health-Monitors über die \ac{api} zugängig gemacht werden, müssen diese in der in Listing \ref{lst:InitTargets} aufgeführten \textit{InitTargets()} Funktion hinzugefügt werden. 
\begin{lstlisting}[caption={InitTargets Funktino der DataBaseAPI}, label={lst:InitTargets}]
    private void InitTargets()
    {
        Targets = new List<Target>();
        switch (config.deviceType)
        {
            case "FLX":
                Targets.Add(new Target("16777217", "Temperature 1", "TempReadings"));
                ...
                break;
            case "OfficePC":
                Targets.Add(new Target("16777470", "CPU Package", "TempReadings"));
                ...
                break;
        }
        Targets.Add(new Target("latest", "CurrentSystemStatus", "SystemStatus"));
        Targets.Add(new Target("latest", "SystemStatusHistory", "HistorySystemStatus"));
        Targets.Add(new Target("latest", "SystemReliability", "SystemReliability"));

    }
\end{lstlisting}