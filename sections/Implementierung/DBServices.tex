\newpage
\section{HM.DBServices}\label{sec:HMDBServices}
Das \textit{HM.DBServices} Verzeichnis Implementiert die alle Klassen und Funkitonen, welche mit externen Dateien interagieren. Neben der \textit{SQLiteWrapper}- und \textit{ConfigurationReader}-Klasse, organisiert das Verzeichnis die Dateien \textit{ConfigFile.json} und \textit{HealthMonitoringDataBase.sql}.

\subsection*{ConfigurationReader}
Die \textit{ConfigurationReader}-Klasse wird benötigt, um die Konfigurationsdatei der Anwendung auslesen zu können. In dieser müssen neben der Hardwarekonfiguration und dem Namen der Datenbankdatei, die Intervalle der einzelnen Jobs festgelegt werden. Zuletzt wird die IP Addresse der Zielplattform hinnterlegt um die \ac{api} von von einem externen Rechnern erreichen zu können.\\
Hierzu wird zunächst eine \ac{json} Datei angelegt, welche folgende Struktur hat.
\begin{lstlisting}[caption={Konfigurationsdatei der Hardware-Health-Monitor Anwendung}, label={lst:ConfigFileStrukture}]
{
  "deviceType": "OfficePC",
  "databaseFileName": "Finaltesting.db",
  "readingJobInterval": 3,
  "currentSystemStatusJobInterval": 60,
  "systemStatusHistoryJobInterval": 300,
  "systemIP": "192.168.0.100"
}
\end{lstlisting}
Die \textit{Configu}

Desweitern wird die \textit{Configuration}-Klasse angelegt, welche der Struktur der \textit{ConfigFile.json} Datei folgt. Anschließend wird in der \textit{ConfigurationReader}-Klasse, über den \textit{JsonSerializer} die  Konfigurationsdatei in das \textit{Configuration} Objekt übernommen. 