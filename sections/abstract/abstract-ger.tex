% !TeX root = ../../template.tex
\chapter*{Abstract}
Die Firma \ac{p+f} ist im Bereich der Prozessautomation führender Hersteller für industrielle Sicherheitsausstattung. Das Produktportfolio umfasst eine Reihe industrieller Computersysteme für den Einsatz in explosionsgeschützten Bereichen. Durch den Einsatz in industriellen Umgebungen, sind diese Systeme Umwelteinflüssen wie Sonneneinstrahlung oder Vibrationen ausgesetzt. Durch eine falsche Abschätzung dieser Umwelteinflüsse können schädliche und unzulässige Betriebe entstehen. Oftmals können diese Betriebe nicht wahrgenommen werden. Um diesem Problem entgegenzuwirken, soll die in den Systemen verbaute Sensorik verwendet werden, um auf solche Betriebe hinzuweisen.\\
Diese Arbeit befasst sich im ersten Teil mit der Erarbeitung eines Konzepts, so wie einer Architektur für eine solche Healthmonitoring Lösung für die \acl{p+f} \ac{hmi}-Plattformen. Dabei wird eine Architektur entwickelt, welche es ermöglicht plattformunabhängig Sensorwerte auszulesen und zu speichern. Des Weiteren wird ein Modell zur Zustandsbewertung der Plattformen definiert.\\
Im zweiten Teil dieser Arbeit wird ein Prototyp der Anwendung implementiert. Dieser liest Sensoren der Plattformen aus, bewertet anhand der ausgelesenen Daten den Zustand der Geräte und stellt die Daten einer Netzwerk-Schnittstelle bereit.
Zur Visualisierung der Daten wir zudem ein Dashboard erstellt, welches dezentral auf einem separaten Rechner bereitgestellt wird. In diesem können Sensordaten der Plattformen in Echtzeit abgerufen und präsentiert werden. 